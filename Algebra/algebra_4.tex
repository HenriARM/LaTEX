\documentclass[12pt]{article}
\usepackage[utf8]{inputenc}

% math equation and simple text indentation
\usepackage[fleqn]{amsmath} 
\setlength{\mathindent}{0em}
\setlength{\parindent}{0em}
\setlength{\parskip}{1em}

\usepackage { amsmath , amssymb , amsthm, mathtools } %math packages
\usepackage[russian]{babel}
\usepackage{verbatim} %for comments
 
\title{Matricas}
\author{}
\date{}

\begin{document}

\maketitle

\textbf{\underline{Teo.}}  $(AB)^\intercal = B ^\intercal A ^\intercal$
\begin{gather*} 
	\text{vienības matrica } A \cdot E_n = E_n \cdot A = A  \quad det(E_n)=1
\end{gather*}
\textbf{\underline{Teo.}}  $det(AB) = det(A)det(B)$

Ja $det(A)=0$, tad matricu $A$ sauc par \textbf{singulāru} matricu(citos tekstos - par deģenerētu matricu).

Secinājums: reizinājums $ABCD \ldots Z$ ir singulāra matrica tad un tikai tad, ja kaut viena no matricām $A, B, C, D, \ldots Z $ ir singulāra.

%%%%%%%%%%%%%%%%%%%%%%%%%%%%%%%
\pagebreak

Inversā (apgrieztā) matrica $A^1$

$$
A^{-1} \cdot A=A \cdot A^{-1}= E
$$

\begin{gather*}
 	\text{Ja } A  \text{ ir singulāra matrica, tad inversā matrica } A^{-1} \text{ neeksistē.} \\
	\quad \text{jo }  det(A \cdot A^{-1}) = det(E) = 1 = det(A) \cdot det(A^{-1}) \\
	\text{Ja $A$ ir singulāra un $B$ - nesingulāra, tad vienādojumam $A \cdot X = B$ nav atrisinājumu, jo nesingulāru dalīt uz singulāru nevar.}
\end{gather*}

\textbf{\underline{Teo.}} Ja $det(A) \neq 0$ , tad inversā matrica $A^{-1}$ vienmēr eksistē!
 
\begin{gather*}
	\textbf{algebriskais papildinājums } C_{ij} =  (-1)^{i+ j} det(M_{ij}), \\
	\text{\quad kur  $M_{ij}$ ir $(n-1)x(n-1)$ matrica, ko iegūst no matricas $A$, izsvītrojot i-to rindu un j-to kolonu.}\\
	\text{No skaitļiem $C_{ij}$ var sastādīt \textbf{kofaktoru} matricu $A^C  = (C_{ij} | i, j = 1 \ldots n)$},
\end{gather*}

\begin{gather*}
 	\text{Lemma. Reizinot $A \cdot (A^C)^T$ vai $(A^C)^T \cdot A$, iegūst matricu, kurai visi elementi ir 0, }\\ 
 		\text{ \quad izņemot diagonāles elementus, kas ir vienādi ar $det(A)$ } \\
	\text{Secinājums. Ja $det(A) \neq 0$ , tad $A^{-1} = \frac{1}{det(A)} \cdot (A^C)^T $}
\end{gather*}

Gausa-Jordana metode
\begin{gather*}
	TODO
\end{gather*}


%%%%%%%%%%%%%%%%%%%%%%%%%%%%%%%
\pagebreak

Citas teoremas
\begin{gather*}
	(A \cdot B)^{-1} = B^{-1} \cdot A^{-1} \\
	(cA)B = A(cB) = c(AB). \\ %c - skaitlis
	(A+B)^2 =(A+B)\cdot (A+B)= A^2 +AB+BA+B^2 \\
	(A+B)(A-B)=A^2-AB+BA-B^2 \\
	\text{Ja $det(A) \neq 0$, tad vektors $x=A^{-1} \cdot b$ ir sistēmas $A \cdot x = b$ vienīgais atrisinājums.}
\end{gather*}

\text{Piemērs. $3x+2y=1; 4x+3y=1; A=[{3,2},{4,3}]; b={1,1};$} \\
\text{$A^{-1}=[{3,-2},{-4, 3}]; A^{-1}b={1,-1}; x=1; y=-1$.}


\begin{gather*}
\end{gather*}

\begin{comment}
\end{comment}
\end {document}