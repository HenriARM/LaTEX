\documentclass[12pt]{article}
\usepackage[utf8]{inputenc}

% math equation and simple text indentation
\usepackage[fleqn]{amsmath} 
\setlength{\mathindent}{0em}
\setlength{\parindent}{0em}
\setlength{\parskip}{1em}

\usepackage { amsmath , amssymb , amsthm, mathtools } %math packages
\usepackage[russian]{babel}
\usepackage{verbatim} %for comments
 
\title{Kompleksie skaitļi}
\author{}
\date{}

\begin{document}

\maketitle

\begin{gather*}
	\frac{a+ bi}{x+ yi} = \frac{(ax+ by)+ (-ay+ bx)i}{x^2+ y} \\
	i^{-1} = \frac{1}{i} = \frac{i}{i^2} = -i \\ 
	\sqrt{i} = +- \frac{1+i}{\sqrt{2}}
\end{gather*}

Geometriksais veids:
\begin{gather*}
	 z=a+bi;  \quad  r=|z| \quad  \phi =arg(z) \\
	 a= r cos \phi; b= rsin \phi; 
\end{gather*}

Leņķus noasaka, skatoties uz grafiku.
\begin{gather*}
\end{gather*}

\begin{comment}
\end{comment}
\end {document}