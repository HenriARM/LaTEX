\documentclass[12pt]{article}
\usepackage[utf8]{inputenc}

% math equation and simple text indentation
\usepackage[fleqn]{amsmath}
\setlength{\mathindent}{0em}
\setlength{\parindent}{0em}
\setlength{\parskip}{1em}

\usepackage { amsmath , amssymb , amsthm, mathtools } %math packages
\usepackage[russian]{babel}
\usepackage{verbatim} %for comments
 
\title{Polinomi}
\author{}
\date{}

\begin{document}

\maketitle


Iespējami divi polinomu vienādības jēdzieni:

1) Teiksim, ka polinomi $P_1(x)$ un $P_2(x)$ ir \textbf{vienādi kā izteiksmes}, ja to kanoniskais pieraksts satur vienādas $x$ pakāpes un vienādus koeficientus pie attiecīgajām $x$ pakāpēm.

2) Teiksim, ka polinomi $P_1(x)$ un $P_2(x)$ ir \textbf{vienādi kā funkcijas}, ja visiem kompleksiem skaitļiem x, P1(x) = P2(x).

Polinomam ar veseliem koeficientiem var būt dalītāji pat ar kompleksiem koeficientiem, piemēram:

$$
x^2 + 1 = (x - i) \cdot (x + i)
$$

LKD

Divu polinomu LKD, jeb GCD, (greatest common divisor) ir kopīgais dalītājs ar visaugstāko pakāpi.

Ja $D(x)$ ir divu polinomu kopīgs dalītājs, tad tāds ir arī $c \cdot D(x)$. Tāpēc būs lietderīgi vienoties, ka LKD koeficients ir $1$.

Piemērs. $LKD(2x+2, 3x+3) = x+1$, nevis $5x+5$


%%%%%%%%%%%%%%%%%%%%%%%%%%%%%%%%%%%%%%%%%%%%
\pagebreak


\begin{gather}
\end{gather}
\end {document}