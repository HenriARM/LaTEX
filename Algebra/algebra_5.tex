\documentclass[12pt]{article}
\usepackage[utf8]{inputenc}

% math equation and simple text indentation
\usepackage[fleqn]{amsmath}
\setlength{\mathindent}{0em}
\setlength{\parindent}{0em}
\setlength{\parskip}{1em}

\usepackage { amsmath , amssymb , amsthm, mathtools } %math packages
\usepackage[russian]{babel}
\usepackage{verbatim} %for comments
 
\title{Polinomi}
\author{}
\date{}

\begin{document}

\maketitle


Iespējami divi polinomu vienādības jēdzieni:

1) Teiksim, ka polinomi $P_1(x)$ un $P_2(x)$ ir \textbf{vienādi kā izteiksmes}, ja to kanoniskais pieraksts satur vienādas $x$ pakāpes un vienādus koeficientus pie attiecīgajām $x$ pakāpēm.

2) Teiksim, ka polinomi $P_1(x)$ un $P_2(x)$ ir \textbf{vienādi kā funkcijas}, ja visiem kompleksiem skaitļiem x, $P_1(x) = P_2(x)$.

Polinomam ar veseliem koeficientiem var būt dalītāji pat ar kompleksiem koeficientiem, piemēram:

$$
x^2 + 1 = (x - i) \cdot (x + i)
$$

%%%%%%%%%%%%%%%%%%%%%%%%%%%%%%%%%%%%%%%%%%%%
\pagebreak

\textbf{\underline{Teo.}}  Jebkuriem diviem polinomiem $P(x)$ un $Q(x)$, ja $Q$ nav nulles polinoms, tad var atrast vienīgos polinomus – dalījumu $D(x)$ un atlikumu $R(x)$ tādus, ka
$P(x)=Q(x)D(x)+R(x)$, un $R(x)$ pakāpe ir mazāka par dalītāja $Q(x)$ pakāpi vai arī $R(x)$ ir nulles polinoms.

%atrast pieradijumu tai teoremI
% on 11th slide showing how to recursiveky delete p(x) on q(x) 

\textbf{\underline{Teo.}} Jebkurš polinoms dalās ar jebkuru null-tās pakāpes polinomu.

Piemērs: $x^2+1$ dalās ar $17$.

%%%%%%%%%%%%%%%%%%%%%%%%%%%%%%%%%%%%%%%%%%%%
\pagebreak

Polinomu LKD

Divu polinomu LKD, jeb GCD (greatest common divisor),  ir kopīgais dalītājs ar \textbf{visaugstāko pakāpi}.

\textbf{\underline{Teo.}} Ja $D(x)$ ir divu polinomu kopīgs dalītājs, tad tāds ir arī $c \cdot D(x)$. Tāpēc būs lietderīgi vienoties, ka LKD koeficients ir $1$. \\
Piemērs. $LKD(2x+2, 3x+3) = x+1$, nevis $5x+5$ \\

\textbf{\underline{Teo.}}  $ LKD(P(x), Q(x)) = LKD(P(x) - D(x) \cdot Q(x), Q(x)) $ \\
\textbf{\underline{Pier.}} Ja $P$ un $Q$ abi dalās ar $R$, tad arī $D \cdot Q$ un $P - D \cdot Q$
dalās ar $R$. \\

Tātad LKD mēs varētu meklēt, izmantojot Gausa metodes ideju: \\
\textbf{\underline{Piem.}}  $P(x)=x^3-6x^2+11x-6; \quad Q(x)=x^2-7x+10;  \quad deg(P(x) + Q(x)) = 5$
\begin{gather*}
	\text{1) likvidējam } x^3: \quad P_1=P - x \cdot Q = x^2+x-6 \quad deg(P_1(x) + Q(x)) = 4 \\
	\text{2) likvidējam vienu no } x^2: \quad P_2 = P_1 - Q = 8x - 16 \quad deg(P_2(x) + Q(x)) = 3\\
	\text{3) likvidējam atlikušo } x^2: \quad Q_1 = Q - \frac{x}{8} \cdot P_2 = 10 - 5x \\
	\text{4) likvidējam vienu no x-iem: } \quad Q_2 = Q_1 + \frac{5}{8} \cdot P_2 = 0 \\
	LKD(8x -16, 0) = 8x - 16 \Rightarrow LKD(x^3 - 6x^2+11x-6, x^2-7x+10) = x-2.
\end{gather*}

Ja šis process beigtos ar pāri, kas sastāv tikai no skaitļiem. Tad LKD ir null-tās pakāpes polinoms :  $LKD(P, Q)=1$

Ja P un Q abi ir nulles polinomi, tad uzdevumam nav atrisinājuma (nulles polinoms
dalās ar jebkuru citu polinomu).
Ja viens no polinomiem ir nulles polinoms, bet otrs – nav, tad šis otrs, dalīts ar
vecākās x pakāpes koeficientu, arī ir meklētais LKD, un to arī izdosim kā rezultātu. 

% read till end recursive algo for it

%%%%%%%%%%%%%%%%%%%%%%%%%%%%%%%%%%%%%%%%%%%%
\pagebreak
Polinoma saknes

Polinoma P(x) sakne ir skaitlis c tāds, ka P(c)=0.

Skaitlis c tad un tikai tad ir polinoma
P(x) sakne, ja P(x) dalās ar x−c.
Pierādījums.
a) Ja P(x)=Q(x)(x−c), tad P(c)=0.

\begin{gather*}
\end{gather*}
\end {document}