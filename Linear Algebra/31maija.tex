\documentclass[12pt]{article}
\usepackage[utf8]{inputenc}

% math equation and simple text indentation
\usepackage[fleqn]{amsmath}
\setlength{\mathindent}{0em}
\setlength{\parindent}{0em}
\setlength{\parskip}{1em}

% for big chi
\usepackage{graphicx}% http://ctan.org/pkg/graphicx

\usepackage { amsmath , amssymb , amsthm, mathtools } %math packages
\usepackage[russian]{babel}
\usepackage{verbatim} %for comments

\newcommand*{\bigchi}{\mbox{\Large$\chi$}}% big chi
 
\title{Unitārās telpas}
\author{Henrik Gabrielyan}
\date{the 30 of May 2019}

\begin{document}

\maketitle

\begin{align*}
	R &  & C  & & \\
	\\
 	x_e^{\intercal} \cdot F_e \cdot y_e       & &      f(x,y) = \sum_{i = 1}^{n}  \sum_{j = 1}^{n}  a_{ij} \cdot \overline{x}_i \cdot y_j = x^{*}_e \cdot F_e \cdot y_e & & \\
	& & 	 A^{*} = \overline{A^{\intercal}} & &\\
	\\
 	\text{Simetriska matrica : } a_{ij} = a_{ji}    & &    \text{Ermita(Hermit) matrica : } a_{ij} = \overline{a}_{ji}  & & \\
	 A^{\intercal} = A     & &    A^* = A  & & \\
 	& &    F(x) = f(x,x) \in R & & \\
	& &    (x,y) = \overline{x}_1 \cdot y_1 + \overline{x}_2 \cdot y_2 + \ldots + \overline{x}_n \cdot y_n = x^* \cdot y & & \\
	\\ 
 	\text{Ortogonāls operators}    & &    \text{Unitārs operators : } & &\\
	 A^{\intercal} = A^{-1}    & &    (x,y) = (A(x), A(y)) & & \\
	 \\
 	\text{Simetrisks operators}    & &    \text{Ermita operators} & &\\
 	(A(x),y) = (x,A(y))    & &    (A(x),y) = (x,A(y)) & &\\
\end{align*}

Ermita operatoram visas īpašvērtības ir reālas, eksistē ortonormēta īpašvektoru bāze. \\

\begin{gather*}
	C \\
	A : A^{*} = A \\
	\forall_{x,y} ((Ax,y) = (x,Ay)) \\
	\lambda \text{ - A īpašvertība}, x \text{  -  īpasvektors} \\
	\\
	A \cdot x = \lambda \cdot x \\
	\overline{A \cdot x} = \overline{\lambda \cdot x} \\
	\overline{A} \cdot \overline{x}  = \overline{\lambda} \cdot \overline{x} \\
	(\overline{A} \cdot \overline{x})^{\intercal}  = (\overline{\lambda} \cdot \overline{x})^{\intercal} \\
	x^{*} \cdot A^{*} = \overline{\lambda} \cdot x^{*} \\ 
	\\
	x^* \cdot A \cdot x = x^* \cdot (A \cdot x) = x^* \cdot (\lambda \cdot x) = \lambda \cdot x^* \cdot x = \lambda \cdot |x|^2 \\
	\\ \; \; \| \\ %change
	x^* \cdot A \cdot x = (x^* \cdot A^*) \cdot x = (\overline{\lambda} \cdot x^*) \cdot x = \overline\lambda \cdot x^* \cdot x = \overline\lambda \cdot |x|^2 \\
	\lambda = \overline\lambda, \text{tātad } \lambda \in R \\
\end{gather*}

\newcommand\x{\times}
\newcommand\bigzero{\makebox(0,0){\text{\huge0}}}
\newcommand*{\bord}{\multicolumn{1}{c|}{}}

\begin{gather}
	A \rightarrow  	
	\begin{pmatrix}
	1 & && \bigzero   \\ 
    	& 1 &  \\
	& &  \ddots  &\\ 
        & \bigzero  &  \\
    	& &  & &  1  \\ 	
    \end{pmatrix} \\
     D = P_{s'f}^{-1} \cdot A_{s's} \cdot P_{se} \\    
     A_{s's} = P_{s'f} \cdot D \cdot P_{se}^{-1}
\end{gather}

\pagebreak

 %%%%%%%%%%%%%%%%%%%%%%%%%%%%%%%%%%%%%%%%%%%% %%
 
 Singulārvērtību izjaukums (dekompozīcija) \\
 SVD \\
 
Singular value decomposition \\

Dots: lin. operators $A : L_1 \rightarrow L_2 , \; \; L_1, L_2$ - Eiklīda vai unitāras \\

Teorēma (SVD): katrai reālai (kompleksai) mxn matricai A eksistē tādas ortogonālas (unitāras) matricas U un V un tāda  diagonālmatrica $\sum$, ka

\begin{gather*}
 A = U \cdot \Sigma \cdot V^{*} \\
 \text{m x n = m x m * m x n * n x n }\\
 \Sigma   = 
 \begin{pmatrix}
 	\sigma_1 & && \bigzero   \\ 
    	& \sigma_2 &  \\
	& &  \ddots  &\\ 
        & \bigzero  &  \\
    	& &  & &  \sigma_m  \\ 
 \end{pmatrix} \\
 \sigma_1 \geq \sigma_2 \geq \ldots  \geq \sigma_m \geq 0\\
 \end{gather*}
 
Pozitīvās vērtības $\sigma_i$ sauc par operatora $A$ singulārvērtībām, matricas $U$ i-to stabiņu sauc par kreiso singulārvektoru, kas atbilst $\sigma_i$, matricas $V$ i-to stabiņu sauc par labējo singulārvektoru, kas atbilst $\sigma_i$\\

\pagebreak

%%%%%%%%%%%%%%%%%%%%%%%%%%%%%%%%%%%%%%%%%%%%%%
 
Pier. 
\begin{gather*}
A^{*} \cdot A  - \text{simetriska (Ermita)} \\
(A^{*} \cdot A)^{*} = A^{*} \cdot A^{**} = A^{*} \cdot A \\
 A^{*} \cdot A - \text{nenegatīvi noteikta} \\
x^{*} \cdot A^{*} \cdot A \cdot x = (A \cdot x)^{*} \cdot (A \cdot x) = (Ax, Ax) = |A \cdot x|^{2} \geq 0 \\
A^* A = \underset{n \times n}{V} \cdot D \cdot \underset{n \times n}{V^*} = V \cdot \Sigma \cdot \Sigma ^ * \cdot V^* \\ 
\\ \\
\underset{n \times n}{D} = 
  \begin{pmatrix}
   	d_1 & && \bigzero   \\
	& &  \ddots  &\\ 
        & \bigzero  &  \\
    	& &  & &  d_n  \\ 
 \end{pmatrix} \\ 
 \\ \\
 d_i \in R \\
 d_1 \geq 0 \\
 d_1 \geq \ldots \geq d_n \\
 \sigma_i = \sqrt{d_i} \\ 
 \\ \\
 D = S^2 = S^{\intercal} \cdot S = S^{*} \cdot S = \Sigma^* \cdot \Sigma = \Sigma \cdot \Sigma^*, \\
 \\ \\
 S  =
   \begin{pmatrix}
   	\sigma_1 & && \bigzero   \\
	& &  \ddots  &\\ 
        & \bigzero  &  \\
    	& &  & &  \sigma_n  \\ 
 \end{pmatrix} \\ \\
\end{gather*}

\pagebreak

%%%%%%%%%%%%%%%%%%%%%%%%%%%%%%%%%%%%%%%%%%%%%%

1. Izrēķina $A^* \cdot A$ \\
2. Atrod $A^* \cdot A$ īpašvērtībs, dabū $D$, $\Sigma$ \\
3. Atrod atbilstošas īpašvektorus (ortonormētu sistēmu), dabū $V$ (stabiņi-īpašvektori) \\
 
\begin{gather*}
	A = 
	\begin{pmatrix}
		1 & -1 \\
		-2 & 2 \\
		2 & -2 
	\end{pmatrix}
\end{gather*} 

\begin{gather*}
	A^\intercal = 
	\begin{pmatrix}
		1 & -2 & 2 \\
		-1 & 2 & -2\\
	\end{pmatrix}
\end{gather*} 
 
 \begin{gather*}
 	\begin{cases}
		(u_1,x) = 0 \\
		\ldots \ldots \ldots \\
		(u_r,x) = 0
	\end{cases}
\end{gather*} 

\begin{gather}
	A^\intercal \cdot A = 
	\begin{pmatrix}
		9 & -9 \\
		-9 & 9
	\end{pmatrix}
\end{gather}

\begin{gather}
	\bigchi_{A^\intercal \cdot A } (\lambda) = \lambda^2 - 18\lambda \\
	\lambda_1 = 18, \lambda_2 = 0 \\
	D = 
	\begin{pmatrix}
		18 & 0 \\
		0 & 0
	\end{pmatrix}
\end{gather}

FAS + Grama - Šmita process

\pagebreak

%%%%%%%%%%%%%%%%%%%%%%%%%%%%%%%%%%%%%%%%%%%%%%


\begin{gather*}
	\Sigma =
	\begin{pmatrix}
		3\sqrt2 & 0 \\
		0 & 0
	\end{pmatrix}
\end{gather*}

\begin{gather*}
	V =
	\begin{pmatrix}
		1\sqrt2 & 1\sqrt2 \\
		-1\sqrt2 & 1\sqrt2
	\end{pmatrix}
\end{gather*}

\begin{gather*}
	\lambda_1 = 18 \ldots v_1 =
	\begin{pmatrix}
		1\sqrt2 \\
		-1\sqrt2
	\end{pmatrix}
	\\
	\lambda_2 = 0 \ldots v_2 =
	\begin{pmatrix}
		1\sqrt2 \\
		1\sqrt2
	\end{pmatrix}
\end{gather*}

\begin{gather*}
	u_1 = \frac{1}{3\sqrt2} \cdot
	\begin{pmatrix}
		1 & -1 \\
		-2 & 2 \\
		2 & -2 
	\end{pmatrix}
	\cdot
	\begin{pmatrix}
		1\sqrt2 \\
		-1\sqrt2
	\end{pmatrix}
	=
	\begin{pmatrix}
		1/3 \\
		-2/3 \\
		2/3
	\end{pmatrix}
\end{gather*}

\begin{gather}
	u_i  = \frac{1}{\sigma_i} \cdot A \cdot v_i \\
	\text{līdz pēdējam nenulles} \quad \sigma_i \\
	\text{dabūjam} \quad u_1, u_2, \ldots, u_r \\
	u_{r+1}, \ldots, u_m - \text{papildinām līdz ortonormētai bāzei}
\end{gather}

\pagebreak

%%%%%%%%%%%%%%%%%%%%%%%%%%%%%%%%%%%%%%%%%%%%%%

\begin{gather}
	U =
	\begin{pmatrix}
		1/3 & 2/\sqrt5 & -2/3\sqrt5 \\
		-2/3 & 1/\sqrt5 & 4/3\sqrt5 \\
		2/3 & 0 & 5/3\sqrt5
	\end{pmatrix}
\end{gather}


\begin{gather}
	\begin{pmatrix}
		1/3 & -2/3 & 2/3
	\end{pmatrix}
	\thicksim
	\begin{pmatrix}
		1 & -2 & 2
	\end{pmatrix}
	\\ \\
	FAS: (2; 1; 0) \\
	\quad \quad \; \; (-2; 0; 1) %refactor
\end{gather}

\begin{gather}
	(-2; 0; 1) - \frac{(-2) \cdot 2 + 0 \cdot 1 + 1 \cdot 0}{2^2 + 1^2 + 0} \cdot (2; 1; 0) = \\
	= (-2; 0; 1) + \frac{4}{5} \cdot (2; 1; 0) = \\
	= (-2/5; 4/5; 1) \thicksim (-2; 4; 5)
\end{gather}

\begin{comment}
\begin{gather}
	\begin{pmatrix}
	\end{pmatrix}
\end{gather}

\begin{gather}
\end{gather}
\end{comment}

\end {document}