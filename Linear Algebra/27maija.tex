\documentclass[12pt]{article}
\usepackage[utf8]{inputenc}

% math equation and simple text indentation
\usepackage[fleqn]{amsmath}
\setlength{\mathindent}{0em}
\setlength{\parindent}{0em}
\setlength{\parskip}{1em}

\usepackage { amsmath , amssymb , amsthm, mathtools } %math packages
\usepackage[russian]{babel}
\usepackage{verbatim} %for comments
 
\usepackage{xcolor}
 
\title{Eiklīda telpa}
\author{Henrik Gabrielyan}
\date{the 27 of May 2019}

\begin{document}

\maketitle

\textbf{\underline{Pier.}}  $E = A^{\intercal} \cdot A$
\begin{gather*}
	(x,y) = x_1y_1 + \ldots + x_ny_n  \\
	(x,y) = (A(x), A(y)) \\
	\\
	A \text{ - ortogonāls } \Leftrightarrow A^{\intercal} \cdot A = E \\ 
	A \cdot A^{\intercal} = E \\
	A^{-1} = A^{\intercal} \\
	\\
	f(x,y) = x_e^{\intercal} \cdot F_e \cdot y_e \\
	(x,y) = x_e^{\intercal} \cdot E \cdot y_e = x_e^{\intercal}  \cdot y_e \\
	x^{\intercal} \cdot y = (A \cdot x)^{\intercal} \cdot (A \cdot y) \\
	x^{\intercal} \cdot E \cdot y  = x^{\intercal} \cdot (A^{\intercal} \cdot A) \cdot y \\
	\quad \Downarrow \\
	E = A^{\intercal} \cdot A
\end{gather*}

%%%%%%%%%%%%%%%%%%%%%%%%%%%%%%%%%%%%%%%%%%%%
\pagebreak

\begin{gather*}
	\text{\color{red}{What for is this here?}}\\
	A_g = P_{eg}^{-1} \cdot A_e \cdot P_eg \\
	F_g = P_{eg}^{\intercal} \cdot F_e \cdot P_{eg}
\end{gather*}

\textbf{\underline{T.1}} Katrai kvadrātiskai formai F Eiklīda telpā eksistē ortonormēta bāze e, tāda ka $F_e$ ir diagonālmatrica. \\
\textbf{\underline{T.2}} Jebkuru kvadrātisku formu ar ortogonāliem pārveidojumiem var novest kanoniskājā formā. \\ %ar skalaro reizinajumu var paradit vai ir kv. forma pozitivi noteikta
\textbf{\underline{T.3}}  Ja F un G ir kvadrātiskas formas, un vismaz viena no tām ir pozitīvi noteikta, tad eksistē bāze, kurā abas šīs formas ir kanoniskajā formā. \\
\textbf{\underline{T.4}}  Katram simetriskam operatoram Eiklīda telpā eksistē ortonormēta bāze, kurā tā matrica ir diagonālmatrica. \\

\textbf{\underline{Def.}} Linēaru operatoru A Eiklīda telpā E sauc par simetrisku $\Leftrightarrow$
$$
\forall x,y \in E ( \; (x,A(y))  =  (A(x),y) \;)
$$

\textbf{\underline{Def.}} Simetriska operatora matricu ortonormētā bāzē sauc par simetrisku matricu.
Īpašības : \\
* A - simetriska  $\Leftrightarrow$ $A^{\intercal}$ = A \\
* A = $(a_{ij})$ - simetriska $\Leftrightarrow \forall i,j (a_{ij} = a_{ji})$  \\

\begin{gather*}
	\forall x \forall y \\
	\quad \quad x^{\intercal} (A \cdot y) = (A \cdot x)^{\intercal} \cdot y \\
	\quad \quad x^{\intercal} \cdot A \cdot y = x^{\intercal} \cdot A^{\intercal} \cdot y \\
	\text{tātad } A = A^{\intercal}
	\\
	F_e = P_{se} \cdot F_s \cdot P_{se}  = P_{se}^{-1} \cdot F_s \cdot P_{se}
\end{gather*}

%%%%%%%%%%%%%%%%%%%%%%%%%%%%%%%%%%%%%%%%%%%%
\pagebreak

\textbf{\underline{T.4 Pier: }}  visas $n$ īpašvērtības ir reālas - atliekam. No īpašvektoriem var izveidot ortonormētu bāzi: ar mat. indukciju. \\

Bāze: n = 1 - triviāla: $\lambda_1$ - īpašvērtība, $x \neq 0$ - īpašvektors \\

$\frac{x}{|x|}$ - ortonormēta bāze.  Operatora $A$ matrica - 1 x 1, tā %change 1 x 1

Pāreja: pieņemam, ka dimensijā līdz $n - 1$ esam pierādījuši. \\
\begin{gather*}
	L^{'} = \{e_1\}^\bot. \quad L^{'} \text{ ir invarianta apakštelpa} \\
	x \in L^{'} \Leftrightarrow (x, e_1) = 0 \\
	A(x) \in ? L^{'} : \quad (A(x), e_1) = (x, A(e_1)) = (x, \lambda_1 \cdot e_1) = \lambda_1 \cdot (x, e_1) =  \lambda_1 \cdot 0 = 0 \\ %questin mark over \in
	\\
	E = span(e_1) \oplus L^{'} \\
	A = A_1 \oplus A^{'} \\
	\\
	dimL^{'} = n - 1 \quad  \text{ Pēc ind. pieņ. } L^{'} \text{ eksistē ortonormēta īpašvektoru bāze } e_2, \ldots, e_n \\
	\text{Pielikam klāt } e_1, \text{ dabūjam ortnormētu īpašvektoru bāzi telpā } E
\end{gather*}


%%%%%%%%%%%%%%%%%%%%%%%%%%%%%%%%%%%%%%%%%%%%
\pagebreak

1585. - 1587. \\
1586. Ortonormētā bāzē sim. operatoram / kvadrātiskai formai ir šāda matrica. Atrast ortonormētu bāzi, kurā tam/ tai ir diagonālmatrica.

\begin{gather*}
	A = 
	\begin{pmatrix}
   		17 & -8 & 4 \\
   		-8 & 17 & -4 \\
   		4, & -4 & 11 \\
	\end{pmatrix}
\end {gather*}


1) Atrast īpašvērtības, ieskaitot to kārtas.
$$
\lambda_{1,2} = 9 \\
\lambda_{3} = 27
$$

2) Atrast īpašvektorus. \\
3) Ortogonalizēt vienas un tās pašas īpašvērtības īpašvektorus. \\
4) Normēt.

ipasvektrous meklejam 

%check roberto' s notes
\begin{gather*}
	\begin{pmatrix} %add vertical line before zeros
		8 $ -8 $ 4  | 0 \\
		-8 $ 8 $ -4 | 0 \\
		4 $ -4 $ 2 | 0 \\
	\end{pmatrix} 
% add this symbol ~
	\begin{pmatrix}
		8 $ -8 $ 4  | 0 \\
		0 $ 0 $ 0 | 0 \\
		0 $ 0 $ 0 | 0 \\
	\end{pmatrix}
% add this symbol ~, plus add which vectors are dipendent and independent with tick
	\begin{pmatrix}
		2 $ -2 $ 1  | 0 \\
		0 $ 0 $ 0 | 0 \\
		0 $ 0 $ 0 | 0 \\
	\end{pmatrix} % how to add all actions with matrix beside it?
	\\
	\\
	\begin{pmatrix}
		-10 & -8 & 4 & | & 0 \\
		-8 & -10 & -4 & | & 0 \\
		4& -4 & -16 & | & 0 \\
	\end{pmatrix}
	\begin{pmatrix}
		1 & -1 & -4 & | & 0 \\
		-4 & -5 & -2 & | & 0 \\
		-5 & -4 & 2 & | & 0 \\
	\end{pmatrix}
	\begin{pmatrix}
		1 & -1 & -4 & | & 0 \\
		0 & -9 & -18 & | & 0 \\
		0 & -9 & -18 & | & 0 \\
	\end{pmatrix}
	\begin{pmatrix}
		1 & -1 & -4 & | & 0 \\
		0 & 1 & 2 & | & 0 \\
		0 & 0 & 0 & | & 0 \\
	\end{pmatrix}
\end{gather*}

\begin{gather*}
	FAS: %indent
	(2; -2; 1) \\
	(1;1;0) = e_1 \\
	(-1/2; 0; 1) ~ (-1;0;2) = e_2% add tilde
\end{gather*}

\begin{gather*}
	e_1^{'} = e_1 \\
	e_2^{'} = (-1;0;2) - \frac{(-1) \cdot 1 + 0 + 1 + 2 \cdot 0}{1^2 + 1^2 + 0^2} \cdot (1;1;0) = (-1; 0; 2) - \frac{-1}{2} (1;1;0) = (-1;0;2) + (1/2;1/2;0) = (-1/2; 1/2; 2) ~ (-1;1;4) %tilde
\end{gather*}


\begin{gather*}
	A_{e^{'''}} =
	\begin{pmatrix}
		9 & 0 & 0 \\
		0 & 9 & 0  \\
		0 & 0 & 27 \\
	\end{pmatrix}
\end{gather*}


%%%%%%%%%%%%%%%%%%%%%%%%%%%%%%%%%%%%%%%%%%%%
\pagebreak

\begin{align*}
	R &  & C  & & \\
	\text{Simetriskas bilineāras formas} & & \text{Ermita bilineāras formas} & & \\
	f(x,y) = \sum_{i = 1}^n \sum_{j = 1}^n a_{ij} \cdot x_i \cdot y_j & &  \sum_{i = 1}^n \sum_{j = 1}^n a_{ij} \cdot \overline{x_i} \cdot y_j & & \\ 
	f(x,y) = f(y,x) & & f(x,y) = \overline{f(y,x)} & & \\
	& & f(x, \alpha y) = \alpha \cdot f(x,y) & & \\
	& & f(\alpha \cdot x, y) = \overline{\alpha} \cdot f(x,y) & & \\
	\\
	F(x) = f(x,x) & & F(x) = f(x,x) & & \\
	F_e \text{- simetriska} & & F_e \text{ - Hermitu matrica} & & \\
	F^{\intercal}_e = F_e & & F_e^{*} = F_e & & \\
	& & a_{ji} = \overline{a_{ij}} & & \\
	& & A^{*} = \overline{A^\intercal} & &
\end{align*}



\begin{gather}
\end{gather}
\end {document}