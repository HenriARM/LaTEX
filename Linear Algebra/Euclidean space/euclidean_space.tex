\documentclass[12pt]{article}
\usepackage[utf8]{inputenc}

% math equation and simple text indentation
\usepackage[fleqn]{amsmath}
\setlength{\mathindent}{0em}
\setlength{\parindent}{0em}
\setlength{\parskip}{1em}

\usepackage { amsmath , amssymb , amsthm, mathtools } %math packages
\usepackage[russian]{babel}
\usepackage{verbatim} %for comments
 
\title{Eiklīda telpas}
\author{Henrik Gabrielyan}
\date{the 20 of May 2019}

\begin{document}

\maketitle
 
%%%%%%%%%%%%%%%%%%%%%%%%%%%%%%%%%%%%%%%%%%%
\underline{$1212. - 1216.$}\\

$1212.$  $ 5x_1^2 + x_2^2 +  \alpha x_3^2 + 4x_1x_2-2x_1x_3-2x_2x_3$

\begin{gather*}
F_s = 
\begin{pmatrix}
   5 & 2 & -1 \\
   2 & 1 & -1 \\
   -1 & -1 & \alpha \\
\end{pmatrix}
\end {gather*}

\begin{gather*}
	|5| = 5 > 0 \quad \quad \\ \\
	\begin{vmatrix} 
		5 & 2 \\
		2 & 1 
	\end{vmatrix}
 	= 5 \cdot 1 - 2 \cdot 2  = 5 - 4 - 1 > 0 \\ \\
 	\begin{vmatrix} 
		5 & 2 & -1 \\
		2 & 1 & -1 \\
		-1 & -1 & \alpha
	\end{vmatrix}
	= -
	\begin{vmatrix}
	2 & -1 \\
	1 & -1
	\end{vmatrix}
	+
	\begin{vmatrix}
	5 & -1 \\
	2 & -1  
	\end{vmatrix}
	+ \alpha
	\begin{vmatrix}
	5 & 2 \\
	2 & 1 
	\end{vmatrix} \\ 
	= -(-2 + 1) + (-5 + 2) + \alpha = 1 - 3 + \alpha = \alpha  > 2 
\end{gather*}
\pagebreak

%%%%%%%%%%%%%%%%%%%%%%%%%%%%%%%%%%%%%%%%%%%

Eiklīda telpa

\textbf{\underline{Def.}} Par \textbf{Eiklīda telpu} sauc jebkuru pāri (E,F), kur E ir linēara telpa pār R un F ir pozitīvi noteikta kvadrātiska forma.

F atbilstošo simetrisko bilineāro formu sauc par šīs \textbf{Eiklīda telpas skalāro reizinājumu}. 

x un y skalārais reizniājums : (x, y)

Izvēlamies bāzi, kurā F ir normālformā.

\newcommand\x{\times}
\newcommand\bigzero{\makebox(0,0){\text{\huge0}}}
\newcommand*{\bord}{\multicolumn{1}{c|}{}}

Tajā $F(x)= x_1^2+ \ldots + x_n^2 $
\begin{gather*}
	F_e = 
	\begin{pmatrix}
	1 & && \bigzero   \\ 
    	& 1 &  \\
	& &  \ddots  &\\ 
        & \bigzero  &  \\
    	& &  & &  1  \\ 	
    \end{pmatrix}
\end{gather*}

$(x,y) = x_1y_1 + x_2y_2 + \ldots + x_ny_n$

Skalārā reizinājuma īpašības: \\
* $(x,y) = (y,x)$ \\
* $(x_1+x_2, y) = (x_1,y) + (x_2,y)$ \\
* $(\alpha x,y)= \alpha(x,y)$ \\
* $ x \neq 0 \rightarrow (x,x) > 0$ \\
$(0,0) = 0$ \\

Vektora x garums: \\
$|x| = \sqrt{(x,x)}$ \\
$|x| = \sqrt{x_1^2 = x_2^2 + \ldots + x_n^2}$ \\

Leņķis starp vektoriem x un y \\
$\angle (x,y) = arccos \frac{(x,y)}{|x||y|}$ \\ 
$x \neq 0, y \neq 0$

\pagebreak

Košī nevienādība
\begin{gather*}
|(x,y)| \leq |x| * |y|  \\
  \quad \updownarrow \\
(x,y)^2 \leq |x|^2 |y|^2  \\
 0 \leq (x - \alpha y, x-\alpha y) =  \\
 \; \; \; = (x, x - \alpha y) - \alpha (y, x - \alpha y) = \\
 \; \; \; =  (x,x) - \alpha (x,y) - \alpha (y,x) - \alpha^2 (y,y) = \\
 \; \; \; = (y,y) \alpha^2 - 2 (x,y) \alpha + (x,x) \\
\Rightarrow  y \neq 0
\end{gather*}

%%%%%%%%%%%%%%%%%%%%%%%%%%%%%%%%%%%%%%%%%%%

\textbf{\underline{Def.}} Saka, ka vektori x un y ir ortogonāli $\Leftrightarrow (x,y) = 0$. Apzīmē, kā $ x \bot y$.

\textbf{\underline{Def.}}  Vektoru sistēmu $e_1, e_2, \ldots, e_k$ sauc par ortogonālu   
$\Leftrightarrow  \forall i \forall j (i \neq j  \rightarrow e_i  \bot e_j)$

\textbf{\underline{Teorēma.}} Ortogonāla nenulles vektoru sistēma lineāri neatkarīga \\
$\alpha_1e_1 + \ldots + \alpha_ke_k = 0$ pareizināsim skalāri ar $e_i$ \\
$(e_i, \alpha_1e_1 + \ldots + \alpha_ke_k) = (e_i,0) $ \\
$\sum_{j=1}^{k}\alpha_j(e_i,e_j)=0$

\textbf{\underline{Def.}} Saka, ka vektors x ir normēts  $\Leftrightarrow |x| = 1 $

Vektoru x normēšana: 

$ x \rightarrow \frac{x}{|x|} = \frac{1}{|x|} x $

\textbf{\underline{Def.}} Ortogonālu vektoru sistēmu,  kas sastāv no normētiem vektoriem, sauc par ortonormētu vektroru sistēmu. Ja tā ir lineārās telpas bāze, tad to sauc par ortonormētu bāzi. 

\pagebreak 

%%%%%%%%%%%%%%%%%%%%%%%%%%%%%%%%%%%%%%%%%%%
\textbf{\underline{Teorēma.}} Jebkurā galīgi dimensionālā Eiklīda telpā eksistē ortonormēta bāze. \\
Jebkuru ortonormētu vektoru sistēmu var papildināt līdz ortonormētai bāzei. \\

\textbf{\underline{Pier.}} 
(E,F) eksistē bāze $e_1, \ldots, e_n$, kurā  F ir normālformā: 

\begin{gather*}
	F_e = 
	\begin{pmatrix}
	1 & && \bigzero   \\ 
    	& 1 &  \\
	& &  \ddots  &\\ 
        & \bigzero  &  \\
    	& &  & &  1  \\ 	
    \end{pmatrix}
    = (f_{ij}) \\ \\ 
    f_{ij} = (e_i, e_j) = 
    \begin{cases}
	0, ja \; \;  i \neq j \\
	1, ja \; \; i = j 
     \end{cases}
\end{gather*}


\begin{gather*}
	e_1, \ldots, e_k \\
	dim E = n > k \\
	\begin{cases}
		(x, e_1) = 0 \\ 
		\cdots \cdots \cdots \\ 
 		(x, e_k) = 0
	\end{cases} \\
	k \; vien. \\
	n \; mainiigie \\
	k < n
\end{gather*}

\end {document}
